\begin{thebibliography}{99}

    \bibitem{conagua-2011}
    Comisión Nacional del Agua.
    \textit{Estadísticas del Agua en México, 2011.} Recuperado 8 de agosto de 2021, de \url{http://www.conagua.gob.mx/CONAGUA07/Publicaciones/Publicaciones/SGP-1-11-EAM2011.PDF}
    
    \bibitem{undesa2008}
    United Nations. Department of Economics and Social Affairs, Population Department,
    \textit{World Population Prospects: The 2008 Revision: Medium fertility variant 2010 - 2050.}
    Consultado en: \url{http://esa.un.org/unpd/wpp2008/index.html} 
    
    \bibitem{fondo2021}
    Fondo para la Comunicación y la Educación Ambiental, A.C.
    \textit{Contaminación en México.} Recuperado 10 de agosto de 2021, de \url{https://agua.org.mx/agua-contaminacion-en-mexico/}
    
    \bibitem{fernandez2020}
    Olaiz-Fernández GA, Gómez-Peña EG, Juárez-Flores A, Vicuña-de Anda FJ, Morales-Ríos JE, Carrasco OF. \textit{Panorama histórico de la enfermedad diarreica aguda en México y el futuro de su prevención. Salud Publica Mex [Internet]}. 20 de diciembre de 2019 [citado 19 de agosto de 2021];62(1, ene-feb):25-3. Disponible en: \url{https://saludpublica.mx/index.php/spm/article/view/10002}
    
    \bibitem{soto2016}
    Soto-Estrada, Guadalupe; Moreno-Altamirano, Laura  y  Pahua-D\'iaz, Daniel. \textit{Panorama epidemiológico de México, principales causas de morbilidad y mortalidad.} Rev. Fac. Med. (Méx.) [online]. 2016, vol.59, n.6 [citado  2021-08-19], pp.8-22, ISSN 2448-4865. Disponible en: \url{http://www.scielo.org.mx/scielo.php?script=sci_arttext&pid=S0026-17422016000600008&lng=es&nrm=iso}. 
    
    \bibitem{camara2021}
    Cámara de Diputados. \textit{Constitución Politica de los Estados Unidos Mexicanos.} Artículo 4, consultado en Agosto del 2021 en:  
    \url{http://www3.diputados.gob.mx/camara/001_diputados/012_comisioneslxii/01_ordinarias/002_agua_potable_y_saneamiento/13_marco_juridico/01_constitucion_politica_de_los_estados_unidos_mexicanos}
    
    \bibitem{merida2020}
    Mérida-Cano, Marvin Eduardo.
    \textit{Calidad bacteriológica del agua y su relación con el potencial de óxido reducción (ORP)}, 2020.
    
    \bibitem{astudillo2019}
    Astudillo-Naderas, Ricardo y Montes de Oca Barrera, Fabiola. \textit{Electrodo de grafito como electrodo indicador de reacciones redox}.  Universidad Nacional Autonoma de México, Facultad De estudios Superiores Cuautitlán, 2019.
    
    \bibitem{steininger1996}
    Steininger, Jacques y Pareja, Catherine. 
    \textit{Respuesta Del Sensor De ORP En Agua Clorada,} 1996
    
    \bibitem{lin2017}
    Wen-Chi, Lin; Klaus Brondum, Charles W. Monroe; y Mark A. Burns. \textit{Monitoreo del pH, potencial de oxidación-reducción (ORP) y conductividad de agua,}  2017
    
    \bibitem{trevor2004}
    Suslow, Trevor V. \textit{Potencial de oxidación-reducción (ORP) para el monitoreo, control y documentación de la desinfección del agua}, 2004.
    \end{thebibliography}
    